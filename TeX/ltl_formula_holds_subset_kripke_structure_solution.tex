We assume that a certain LTL formula $\mathbf{A}f$ holds on $M$. Since the
transition relation of $M$ is a superset of the transition relations of $N$,
there is always the possibility to construct a Kripke structure equivalent to
$M$ by extending $N$ with additional transitions/states. Since a LTL formula
quantifies over \emph{all paths} in a Kripke structure the set of all formulas
which hold on $N$ gets smaller when we extend $N$. This means that the set of
all LTL formulas which hold on $N$ is a superset of all the LTL formulas which
hold on $M$. From this follows that, if a certain LTL formula $\mathbf{A}f$
holds on $M$, then this formula also has to hold on $N$.
