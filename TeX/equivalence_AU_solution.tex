We need to prove that $\mathbf{A}(f \mathbf{U} g)$ is equivalent to $¬E(¬g
\mathbf{U} ¬f ∧ ¬g) ∧ ¬\mathbf{EG} ¬g$. We use the following equivalences:

\begin{enumerate}[label=(\arabic*)]
    \item $\mathbf{A} f ⇔ ¬\mathbf{E}¬f$
    \item $¬(f \mathbf{U} g) ⇔ (\mathbf{G}¬g ∨ ¬g \mathbf{U} ¬f ∧ ¬g)$
    The formula above is true since for $f$ until $g$ to not hold:
        \begin{enumerate}

            \item $g$ has to not hold at all (there exists no $k$ such that
            $π^k⊧g$) or

            \item $f$ might hold at first, while $g$ does not hold, but $g$
            does not hold after that ($¬g \mathbf{U} ¬f ∧ ¬g$).

        \end{enumerate}

    \item $\mathbf{E}(f ∨ g) ⇔ \mathbf{E}f ∨ \mathbf{E}g$
    \item $¬ (f ∨ g) ⇔ ¬f ∧ ¬g$
\end{enumerate}

to deduct the following proof:
\begin{align*}
    \mathbf{A}\left(f \mathbf{U} g\right)
    & \stackrel{(1)}{⇔} ¬\mathbf{E} ¬\left(f\mathbf{U} g\right)\\
    & \stackrel{(2)}{⇔} ¬\mathbf{E} \left(\left(\mathbf{G}¬g\right) ∨
      \left(¬g\mathbf{U} ¬f ∧ ¬g\right)\right)\\
    & \stackrel{(3)}{⇔} ¬\left(\mathbf{EG}¬g ∨
      \mathbf{E}\left(¬g\mathbf{U} ¬f ∧ ¬g\right)\right)\\
    & \stackrel{(4)}{⇔} ¬\mathbf{EG}¬g ∧ ¬\mathbf{E}\left(¬g\mathbf{U} ¬f ∧
      ¬g\right)\\
\end{align*}
