One of the standard ways to check an LTL formula is to construct a Büchi
automaton for the negated LTL formula $¬f$ and the given Kripke structure $M$.
These two state machines accept the language $ℒ(¬f)$ respectively $ℒ(M)$. We
now construct a Büchi automaton representing the intersection of the two
languages. If the language accepted by this state machine is empty then $M⊧f$
holds. On the other hand, if there exist infinite words accepted by the
automaton, then these words are counterexamples for $M⊧f$.\\

We now need to show that there exists either a finite path or a finite path
with a loop for the language $ℒ(¬f) ∩ ℒ(M)$ if $M⊭f$ is true. We know that if
there exists a counter-example, then there has to be a path in the Büchi
automaton, where at least one acceptance state occurs infinitely often. This
means that this path has to contain a loop. This implies that there has to be
a finite counterexample for $f$, which includes a
loop~\cite{Norrish2010TemporalLogic}.
