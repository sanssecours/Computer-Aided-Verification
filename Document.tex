%!TEX TS-program = xelatex

% -- Document Class -----------------------------------------------------------

\documentclass[a4paper, 12pt]{article}

% -- Packages -----------------------------------------------------------------

% Language
\usepackage{polyglossia}
    \setmainlanguage{english}
    \setotherlanguage{german}
% Context sensitive quotation
\usepackage{csquotes}
% Citations
% \usepackage[style=alphabetic, backend=biber]{biblatex}
    % \addbibresource{References.bib}
% Extend options for positioning floats
\usepackage{float}
% Support highlighting of certain parts of the text
\usepackage{framed}
% Headers & Footers
\usepackage[automark, nouppercase]{scrpage2}
% To change the format of titles
\usepackage{titlesec}
% Support for unicode math fonts
\usepackage{unicode-math}
% Extended color support
\usepackage[x11names]{xcolor}
% Extras for XƎTEX
\usepackage{xltxtra}
% Hyperlinks and pdf properties
\usepackage{hyperref}

% -- Color Definitions --------------------------------------------------------

% Background color for syntax highlighting
\definecolor{bgcolor}{rgb}      {1,     1,      1}

% Custom color definitions
\definecolor{aqua}{rgb}         {0,     0.56,   1}
\definecolor{bluegray}{rgb}     {0.22,  0.46,   0.84}
\definecolor{grape}{rgb}        {0.56,  0,      1}
\definecolor{orchid}{rgb}       {0.41,  0.13,   0.55}
\definecolor{orange}{rgb}       {1,     0.54,   0}
\definecolor{silver}{rgb}       {0.57,  0.57,   0.57}
\definecolor{turquoise}{rgb}    {0,     0.86,   0.84}

% -- Macros -------------------------------------------------------------------

\newcommand{\Title}{Exercises}
\newcommand{\TitleDescription}{Computer Aided Verification}
\newcommand{\Version}{1}
\newcommand{\Subject}
    {Solutions for the exercises of the course Computer Aided Verification}
\newcommand{\KeyWords}{BDD, Kripke structure}
\newcommand{\LeftFooter}{\Title~—~\TitleDescription}

\newcommand{\AuthorOne}{René Schwaiger}
\newcommand{\MailOne}{\href{mailto:sanssecours@f-m.fm}{sanssecours@f-m.fm}}

% Syntax highlighting definitions
% Text
\newcommand{\hlstd}[1]{\textcolor{black}{#1}}
% Numbers
\newcommand{\hlnum}[1]{\textcolor{DarkOrchid4}{#1}}
\newcommand{\hlesc}[1]{\textcolor[rgb]{1,0,1}{#1}}
% Strings
\newcommand{\hlstr}[1]{\textcolor{SeaGreen3}{#1}}
\newcommand{\hlpps}[1]{\textcolor[rgb]{0.51,0.51,0}{#1}}
\newcommand{\hlslc}[1]{\textcolor[rgb]{0.51,0.51,0.51}{\it{#1}}}
\newcommand{\hlcom}[1]{\textcolor{aqua}{#1}}
\newcommand{\hlppc}[1]{\textcolor[rgb]{0,0.51,0}{#1}}
\newcommand{\hlopt}[1]{\textcolor[rgb]{0,0,0}{#1}}
\newcommand{\hllin}[1]{\textcolor[rgb]{0.33,0.33,0.33}{#1}}
% Keywords
\newcommand{\hlkwa}[1]{\textcolor{DodgerBlue3}{#1}}
\newcommand{\hlkwb}[1]{\textcolor[rgb]{0,0.34,0.68}{#1}}
\newcommand{\hlkwc}[1]{\textcolor{DarkOrchid4}{#1}}
% Functions
\newcommand{\hlkwd}[1]{\textcolor{orange}{#1}}

\newcommand{\codeinput}[1]
{
    \begin{leftbar}
        \input{Code/#1}
    \end{leftbar}
}

\newcommand{\code}[1]
{
    \hl{\texttt{#1}}
}

% -- Document Properties ------------------------------------------------------

% No indendation after paragraph
\setlength\parindent{0cm}

% Hyperref properties
\hypersetup
{
    pdftitle    = {\Title},
    pdfsubject  = {\Subject},
    pdfauthor   = {\AuthorOne},
    pdfkeywords = {\KeyWords},
    colorlinks  = true,
    linkcolor   = black,
    anchorcolor = black,
    citecolor   = silver,
    urlcolor    = orange
}

% -- Fonts --------------------------------------------------------------------

% Use same size for numbers and other text
\defaultfontfeatures{Numbers=Lining}

% Set fonts for document
\setmainfont[Mapping=tex-text]{Avenir Next}
\setsansfont[Mapping=tex-text]{Ubuntu}
\setmonofont[Scale=MatchLowercase]{Menlo}
\setmathfont{Asana-Math.otf}
\setmathfont[range=\mathtt, Scale=MatchLowercase]{Menlo}

% Define font styles
\newfontfamily\Zapfino{Zapfino}

% -- Header And Footers -------------------------------------------------------

% Use normal font instead of italic font for head
\renewcommand{\headfont}{\normalfont}

% Set headers and footers
\ihead{\headmark}
\ohead{}
\ifoot{\LeftFooter}
\ofoot{\thepage}

% Set height of head
\setlength{\headheight}{1.8\baselineskip}

% Set thickness of separation line in header, footer
\setheadsepline{0.5pt}
\setfootsepline{0.5pt}

% -- Titlepage ----------------------------------------------------------------

\begin{document}

\begin{titlepage}

    \begin{center}
        % Title and title-description
        {\Large\Zapfino \Title}
        \vskip 0.5cm
        {\Large\textit\TitleDescription}
        \vskip 1cm
        \hrule
        \vskip 0.5cm
        % Information about author
        \begin{tabular}{p{8cm}l}
            \AuthorOne  & \MailOne\\
        \end{tabular}
        \vskip 0.5cm
        \hrule
        \vskip 13.5cm
    \end{center}

    % Date and version number
    \begin{leftbar}
        \begin{tabular}{ll}
            \textbf{Version}    & \Version\\
            \textbf{Date}       & \today
        \end{tabular}
    \end{leftbar}

\end{titlepage}

% -- Table of Contents --------------------------------------------------------

% Set section format for table of contents
\titleformat{\section}{\sffamily\bfseries}{}{0pt}{}[{\color{aqua}\hrule}]

% Set separation of dots between name of section and page number to such a high
% value that there will be no points in the table of contents
\makeatletter \renewcommand{\@dotsep}{10000} \makeatother
% Use blank header and footer
\pagestyle{empty}
% Start on new page
\newpage
% The table of contents starts at the second page
\setcounter{page}{2}
% Set table of contents
\tableofcontents

% -- Section & Paragraph Style ------------------------------------------------

% Set format for section
\titleformat{\section}
    {\large\sffamily\bfseries}  % Large, bold, sans serif font for section
    {}                          % No format applied to whole title
    {0pt}                       % No separation between label and title
    {\thesection~·~}            % Start with section number
    [{\color{aqua}\hrule}]      % Underline with blue ruler

% Set format for other sections and paragraphs
% Color = orchid, Font = bold, sans serif
\titleformat*{\subsection}{\color{orchid}\sffamily\bfseries}
\titleformat*{\subsubsection}{\color{orchid}\sffamily\bfseries}
\titleformat*{\paragraph}{\color{orchid}\sffamily\bfseries}
\titleformat*{\subparagraph}{\color{orchid}\sffamily\bfseries}

% -- Page Style ---------------------------------------------------------------

% Start with text on a new page
\newpage
% Display headers and footers
\pagestyle{scrheadings}

% -- Text ---------------------------------------------------------------------

\section{Binary Decision Diagrams}

\subsection{Exercise 1}

Give a linear time algorithm for BDD isomorphism as defined on page 9.

\subsection{Exercise 2}

Describe a size-efficient BDD for the relation “a ≥ b” for n-bit integer
numbers.

\subsection{Exercise 3}

Describe an algorithm which transforms a BDD into an equivalent Boolean
formula.

\section{Temporal Logic}

\subsection{Exercise 4}

Prove the equivalence for $A(f U g)$ on page 16.

\subsection{Exercise 5}

Show the following lemma: Let $M$ and $N$ be two Kripke structures such that
the transition relation of $M$ is a superset of the transition relation of
$N$. If an LTL property $f$ holds on $M$, then $f$ also holds on $N$.

\subsection{Exercise 6}

Show that $AFG p$ is not logically equivalent to $AFAG p$.

\subsection{Exercise 7}

Describe a simple model checker for CTL over Kripke structures in pseudocode.

\subsection{Exercise 8}

Find a translation of the $U$ operator to propositional logic in bounded model
checking
\subsection{Exercise 9}

Show that all LTL properties have counterexamples which are either finite
paths or finite paths with a loop. Hint: Use the fact that LTL specifications
can be translated into Buechi automata.

\subsection{Exercise 10}

Give an LTL specification where the smallest counterexample is larger than the
number of states in the Kripke structure.

\subsection{Exercise 11}

Show how you can use SMV to solve chess problems. “Given a chess board, white
has a winning strategy in 3 moves.” How do you describe the board? What is
the specification?

% -- Bibliography -------------------------------------------------------------

% Set section format for bibliography
% \titleformat{\section}{\sffamily\bfseries}{}{0pt}{}[{\color{aqua}\hrule}]
% Display bibliography
% \printbibliography

\end{document}
